\documentclass[12pt,a4paper,oneside]{ctexart}
\CTEXsetup[format={\bfseries}]{section}
\CTEXsetup[format={\small\bfseries}]{subsection}
\usepackage{amsmath,amsthm,amssymb,bm,graphicx,hyperref,mathrsfs,geometry,fancyhdr,booktabs,array,float,multirow,subfigure,tcolorbox,cleveref,algorithm,algorithmicx,algpseudocode}
\renewcommand{\algorithmicrequire}{\textbf{Input:}}  
\renewcommand{\algorithmicensure}{\textbf{Output:}}

\geometry{left=2.50cm,right=2.50cm,top=3.00cm,bottom=3.00cm}
\pagestyle{empty}

\hypersetup{
    colorlinks=true,
    linkcolor=purple,
    filecolor=purple,
    urlcolor=purple,
    citecolor=cyan,
}

\begin{document}

\begin{center}
    \Large \textbf{Shengtang Huang}
\end{center}

\section*{School Address}

\footnotesize

School of the Gifted Young

Univerisity of Science and Technology of China

\section*{Contact}

Gmail: peanuttang1320061044@gmail.com

School Email: peanuttang@mail.ustc.edu.cn

\section*{Research Interest}

Algorithmic graph theory; randomized algorithms and pseudorandomness; combinatorics; operations research.

\section*{Education}

B.S. in Computer Science and Technology, Univerisity of Science and Technology of China (currently a sophomore student, till January 2024)

School of the Gifted Young (supervised by Prof. Xue Chen)

\section*{Publications}

None

\section*{Teaching}

\textbf{Univerisity of Science and Technology of China (TA)}

\begin{tabular}{ll}
    2024 Spring & Introduction to Algorithm (going to be)
\end{tabular}

\section*{Honors and Awards}

\textbf{Univerisity of Science and Technology of China}

\begin{tabular}{ll}
    2023 & Bronze Prize, Outstanding Freshmen/Undergraduates Scholarship.
\end{tabular}

\vspace{0.5cm}

\textbf{International Collegiate Programming Contest}

\begin{tabular}{ll}
    2023 & Sliver Prize, The 2023 ICPC Asia Shenyang Regional Contest. (33rd place) \\
    2022 & Bronze Prize, The 2022 ICPC Asia Jinan Regional Contest. (130th place)
\end{tabular}

\vspace{0.5cm}

\textbf{Olympiad in Informatics}

\begin{tabular}{ll}
    2020 & Sliver Prize, Asia-Pacific Informatics Olympiad \\
    2020, 2021 & First Prize, National Olympiad in Informatics in Provinces
\end{tabular}

\section*{Citizenship}

Chinese (The People's Republic of China).

\section*{Relevant Skills}

Languages: Chinese(native), English(fluent).

\section*{References}

\begin{itemize}
    \item \href{http://staff.ustc.edu.cn/~xuechen1989/}{Xue Chen} (Professor), Department of Computer Science and Technology, Univerisity of Science and Technology of China.
    
          Email: xuechen1989@ustc.edu.cn
\end{itemize}

\end{document}